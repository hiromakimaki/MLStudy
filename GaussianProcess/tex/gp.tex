\documentclass{article}

\usepackage{amsmath}
\usepackage{amsfonts}

\title{Memo for Gaussian Process}
\author{hiro\_makimaki}
\date{2019/04/15}

\begin{document}
\maketitle

\section{Gaussian Process}
\begin{itemize}
  \item $\mathcal{X}$ : input data space
  \item $f: \mathcal{X} \rightarrow \mathbb{R}$ : real-valued function
  \item $\mu: \mathcal{X} \rightarrow \mathbb{R}$
  \item $k: \mathcal{X} \times \mathcal{X} \rightarrow \mathbb{R}$
\end{itemize}

We consider the case that for any natural number $N \in \mathbb{N}$ and inputs $x_{1}, x_{2}, \hdots, x_{N} \in \mathcal{X}$,
the output vector is normally distributed as follows:
\[
  \left( \begin{array}{c} f(x_{1}) \\ f(x_{2}) \\ \vdots \\ f(x_{N}) \end{array} \right)
  \sim N \left(
  \left( \begin{array}{c} \mu(x_{1}) \\ \mu(x_{2}) \\ \vdots \\ \mu(x_{N}) \end{array} \right),
  \left( \begin{array}{cccc} k(x_{1}, x_{1}) & k(x_{1}, x_{2}) & \hdots & k(x_{1}, x_{N}) \\
                                     k(x_{2}, x_{1}) & k(x_{2}, x_{2}) & \hdots & k(x_{2}, x_{N}) \\
                                     \vdots & \vdots & \ddots & \vdots \\
                                     k(x_{N}, x_{1}) & k(x_{N}, x_{2}) & \hdots & k(x_{N}, x_{N}) \end{array} \right)
  \right).
\]
Then, we call $f$ a Gaussian Process and express as follows:
\[ f \sim \text{GP} ( \mu(x), k(x, x')). \]
The function $k$ is called as a kernel function.
\end{document}